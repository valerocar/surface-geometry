\documentclass[11pt]{article}
\usepackage[margin=1in]{geometry}
\usepackage{graphicx}
\usepackage{microtype}
\usepackage{caption}
\title{A Brief History of the Circle}
\author{}
\date{}
\begin{document}
\maketitle
\section*{Introduction}
From Neolithic artisans to modern analysts, the circle has served as a template for tracking the heavens, partitioning land, and modeling periodic phenomena. This note sketches key milestones in how different cultures understood and refined the idea of circularity.
\section*{Ancient Origins}
Early evidence of circular constructions appears in megalithic observatories such as Nabta Playa and Stonehenge, where alignments of posts and stones encoded seasonal cycles. Babylonian scribes later recorded approximations of the circumference-to-diameter ratio, typically taking $\pi \approx 3$, sufficient for surveying and temple design.
\begin{figure}[h]
  \centering
  \includegraphics[width=0.45\textwidth]{figures/unit_circle.png}
  \caption{A geometric reminder of constant radius: the circle as all points equidistant from a center.}
  \label{fig:unitcircle}
\end{figure}
\section*{Classical Greek Formalization}
Greek mathematicians translated empirical rules into deductive geometry. In \emph{Elements}, Euclid codified definitions involving chords, tangents, and inscribed angles, while Archimedes introduced area and circumference bounds via inscribed and circumscribed polygons, squeezing $\pi$ between ever tighter estimates.
\begin{figure}[h]
  \centering
  \includegraphics[width=0.45\textwidth]{figures/polygon_approx.png}
  \caption{Archimedes refined $\pi$ by bracketing a circle with regular polygons.}
  \label{fig:archimedes}
\end{figure}
\section*{Islamic and Medieval Transmission}
Between the 9th and 15th centuries, scholars from Baghdad to Córdoba preserved Greek treatises, expanded trigonometric tables, and developed instruments such as the astrolabe, whose stereographic projection depends on circular arcs. Their commentaries reintroduced rigorous circle theorems to Western Europe.
\section*{Renaissance to Modern Era}
The quest for exact quadrature motivated calculus. Descartes used algebraic coordinates to express circles analytically, and Newton linked circular motion with gravitational attraction. The 19th century proved $\pi$ transcendental, severing hopes of straightedge-and-compass circle squaring, while 20th-century topology generalized circles into higher-dimensional spheres and Lie groups.
\section*{Conclusion}
Although defined simply, the circle binds together astronomy, navigation, analysis, and modern signal processing. Each historical layer, from stone monuments to transcendental proofs, deepened our understanding of symmetry and periodicity. Figures~\ref{fig:unitcircle}--\ref{fig:archimedes} reflect the enduring geometric intuition behind these advances.
\end{document}
